\section{Взаимодействие между ионами в растворе, ионные уравнения реакций. Связывание ионов, направление реакций ионного обмена. Произведение растворимости.}
\subsection{Взаимодействие между ионами в растворе, ионные уравнения реакций.}
\subsubsection{Взамидействие меджде ионами в растворе. Активность.}
При диссоциации в растворах вещества распадаются на ионы, чем больше их оказывается в расстворе, тем силнее их взаимодействие друг с другом. Для более точного описания их поведения следует использовать не их концентрацию, как ранее, а их \textbf{активность} $a$, которая определяется как $\gamma C$, где $C$ -- концентрация, а $\gamma$ - коэффициент активности. Коэффициент активности служит мерой отклонения поведения раствора (или компонента раствора) от идеального (то есть случая, когда активность совпадает с концентрацией). Вообще активность подбирается так, чтобы вид химического потенциала компонента в растворе оставался таким же как для идеального раствора.
\subsubsection{Ионные уравнения реакций}
Сильные электролиты в растворах распадаются на ионы, для уравнений реакций в растворах существует уточняющая запись, отражающая это соображение. Пусть например взаимодействуют два вещества $\mathrm{A}\mathrm{B}$ и $\mathrm{C}\mathrm{D}$, причём при диссоциации в растворе они распадаются на положительные ионы $\mathrm{A}^{+}$, $\mathrm{C}^{+}$ и отрицательные $\mathrm{B}^{-}$, $\mathrm{D}^{-}$, между ними происходит реакция (ионного) обмена, в результате которой получаются вещества $\mathrm{A}\mathrm{D}$ и $\mathrm{C}\mathrm{B}$, причём последнее не диссоциирует, будем символично обозначать это $\downarrow$ (хотя недиссоциирующие вещества не обязательно выпадают осадком, они могут оказаться и в газообразной форме, например). Уравнение этой реакции будет выглядить следующим образом:
\begin{equation*}
    \ce{AB + CD -> AD + CB v}
\end{equation*}
Подчёркивая ионную природу обмена реакцию записывают несколько иначе:
\begin{equation}
    \ce{A^{+} + B^{-} + C^{+} + D^{-} -> A^{+} + D^{-} + CB v}
\end{equation}
Коогда хотят подчеркнуть основную схему реакции (то есть образование соединения из ионов в растворе) уравнение сокращают и справа и с лева на  одни и те же свободные ионы:
\begin{gather*}
\label{eq:ionlong}
   \ce{\cancel{\ce{A^{+}}} + B^{-} + C^{+} + \cancel{\ce{D^{-}}} -> \cancel{\ce{A^{+}}} + \cancel{\ce{D^{-}}} + CB v} \\
   \ce{B^{-} + C^{+}  -> CB v} 
\end{gather*}
То есть принципиальным процессом в этой является образование вещества $CB$ из катиона  $\mathrm{B}^{-}$ и аниона $\mathrm{B}^{+}$. Последнее уравнение называется \textbf{сокращённым ионным уравнением реакции}, а уравнение~\ref{eq:ionlong} называют \textbf{полным ионным}.

Для примера рассмотрим реакцию нейтрализации гидроксида натрия и солной кислоты:
\begin{gather*}
    \ce{NaOH + HCl -> NaCl + H2O}\\
    \ce{Na^{+} + OH^{-} + H^{+} + Cl^{-} -> Na^{+} + Cl^{-} + H2O}\\
    \ce{\cancel{\ce{Na^{+}}} + OH^{-} + H^{+} + \cancel{\ce{Cl^{-}}} -> \cancel{\ce{Na^{+}}} + \cancel{\ce{Cl^{-}}} + H2O}\\
    \ce{OH^{-} + H^{+} -> H2O}
\end{gather*}
Последнее сокращённое уравнение по сути представляет собой схему всех реакций нейтрализации. 
\subsection{Связывание ионов, направление реакций ионного обмена.}
Образование недиссоцирующего вещества в реакциях, описанных выше называется \textbf{связыванием ионов}. Реакции в растворах электролитов идут в сторону связывания ионов. 
\subsection{Произведение растворимости.}
В общем виде константа диссоциации $K_{\mathcal{D}}$:
\begin{gather*}
    \ce{X_{k_{x}}Y_{k_{y}} <--> k_{x}X^{x+} + k_{y}Y^{y-}}\\
    K_{\mathcal{D}} = \frac{\left\lbrace\ce{X}^{x+}\right\rbrace^{\mathrm{k}_{x}}\cdot \left\lbrace\ce{Y}^{y-}\right\rbrace^{\mathrm{k}_{y}}}{\left\lbrace\ce{XY}\right\rbrace}
\end{gather*}
Здесь $\lbrace \ce{A} \rbrace = a_{\ce{A}}$ -- активность \ce{A}. В пределе идельного газа $\left\lbrace\ce{X}^{x+}\right\rbrace\approx \left[\ce{X}^{x+}\right]$, $\left\lbrace\ce{Y}^{y-}\right\rbrace\approx \left[\ce{Y}^{y+}\right]$ и  $\left\lbrace\ce{XY}\right\rbrace \approx 1$ (квадратные скобки обозначают концентрацию). Поэтому в случае идеального раствора используют величину $K_{\mathrm{sp}}$ -- произведение растворимости:
\begin{equation}
K_{\mathrm{sp}} = \left[X^{x+}\right]^{\matrm{k}_{x}}\cdot \left[Y^{y+}\right]^{\matrm{k}_{y}}
\end{equation}
Например для \ce{CaCO3}:
\begin{gather*}
    \ce{CaCO3 <--> Ca^{2+} + CO3^{2-}}\\
    K_{\mathrm{sp}} = \left[\ce{Ca}^{2+}\right]\cdot\left[\ce{CO3}^{2-}\right]
\end{gather*}
