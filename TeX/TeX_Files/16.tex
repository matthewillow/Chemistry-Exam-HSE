\section{Кислотность по Бренстеду, сопряженные кислоты и основания. Вода как кислота и основание. Автоионизация воды, ион гидроксония. pH растворов. Расчет pH растворов слабых кислот и оснований.}

По Бренстеду \textbf{кислоты} --- соединения, которые способны принимать протон, т.е.:

\begin{equation}
	\ce{A H + B <-> A- + H B+}
\end{equation}

Здесь \ce{A H} и \ce{HB+} --- кислоты, \ce{A-} и \ce{B} --- основания. При этом \ce{A H} и \ce{A-}, а также \ce{H B+} и \ce{B} называются \textbf{сопряженными кислотно-основными парами}.


По Бренстеду, вода (\ce{H2O}) может выступать как в роли основания, так и в роли кислоты. Например, при растворении серной кислоты в воде последняя выполняет роль основания:

\begin{equation}
	\ce{H2SO4 + H2O -> H3O+ + HSO4-}
\end{equation}

С переходом протона взаимодействующие соединения поменялись ролями - серная кислота превратилась в сопряженное основание, а вода (основание) - в сопряженную кислоту \ce{H3O+}.

Если же в воде растворяется основание, то она выполняет роль кислоты и в общем случае происходит следующее:

\begin{equation}
	\ce{B + H2O -> HB+ + OH-}
\end{equation}

В силу указанных свойств волна способна к самоионизации:

\begin{equation}
	\ce{H2O + H2O <-> H3O+ + OH-}
\end{equation}

Указанный ион \ce{H3O+} называется \textbf{ионом гидроксония}. Рассмотрим константу равновесия для ионизации воды: