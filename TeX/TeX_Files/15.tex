\section{Кислоты и основания по Аррениусу. Ион гидроксония. Сильные и слабые кислоты и основания. Константы кислотности и основности. Ступенчатая диссоциация на примере фосфорной кислоты.}
	В самом общем определении:
	
	\textbf{Кислота -- вещество которое в реакции отдает ион $H^+$ }
	
	
	\textbf{Основание -- вещество которое в реакции отдает ион $OH^-$}
	
	Это верно для любых растворов и газов. В более узком смысле понимают, что кислота это вещество которое диссоциирует с образованием иона $H^+$ или иона гидроксония $H_3 O^+$ (он очень редкий)
	\begin{equation*}
	\ce{HA + H_2 O <-> A^- + H_3 O^- }
	\end{equation*}
	
	Аналогично для оснований. Только заменить ион водорода на $\ce{OH^-}$
	
	Чтобы понять слабая кислота или основание или нет надо смотреть на константы диссоциации. Напомню, что если есть реакция диссоциации
	\begin{align*}
	A_x B_y = x A + y B
	\end{align*} 
	То константой диссоциации называют
	\begin{align*}
	K = \dfrac{x[A] \cdot y[B]}{[A_x B_y]}
	\end{align*}
	Где в скобках концентрации.
	
	Если $K > 1$ то кислоту или  основание условно считают сильной. И наоборот.
	
	Применительно к кислотам и основаниям константу диссоциации называют константой кислотности и основности соответственно. 
	
	Можно слегка переписать определение. Для разложения 
	\begin{align*}
	\ce{HA <-> H^+ + A^-}
	\end{align*}
	Имеем
	\begin{align}
	K_a = \dfrac{[H^+]^2}{C - [H^+]}
	\end{align}
	Тут мы воспользовались тем, что ионов кислотного остатка и водорода ровно одинаковое количество. И $C$ это сумма ионов вообще. Понятно что она остается неизменной. Величину $pK_a = -\log_{10} K_a$ Называют кислотностью. Для оснований все аналогично, только ионы водорода поменяйте на $OH^-$ Конечно же все эти константы зависят от температуры. Как именно надо смотреть в таблицу.
	
	Диссоциация для многоосновных кислот протекает ступенчато. Например для фосфорной кислоты
	\begin{align*}
	&\ce{H_3PO_4 <-> 	H^+ + H_2PO_4^-} \quad K_1= \dfrac{[H^+][H_2PO_4^{-}]}{[H_3 PO_4]} = 7.25 \cdot 10^{-3} \\
	&\ce{H_2PO_4^- <-> 	H^+ + HPO_4^{2-}} \quad K_2= \dfrac{[H^+][HPO_4^{2-}]}{[H_2 PO_4^{-}]} = 6.31 \cdot 10^{-8}\\
	&\ce{HPO_4^{2-} <-> 	H^+ + PO_4^{3-}} \quad K_3= \dfrac{[H^+][PO_4^{3-}]}{[H PO_4^{2-}]} = 3.98 \cdot 10^{-13}
	\end{align*}
	
	Как видно с каждой ступенью константа на пять порядков меньше. Это более-менее общее свойство. Такое эмпирическое правило называется правилом  Полинга. Произведение всех констант $K_1 K_2 K_3$ соответствует полной диссоциации.  