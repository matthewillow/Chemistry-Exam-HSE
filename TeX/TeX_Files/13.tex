
\section{Электролитическая диссоциация, электролиты и неэлектролиты. Сильные и слабые электролиты. Степень диссоциации, константа диссоциации. Диссоциация кислот, оснований и солей.}

\textbf{Электролитическая диссоциация} --- процесс распада химического соединения на ионы в растворе или расплаве.\\

Степень диссоциации сильно зависит от концентрации раствора - большая концентрация подавляет дисссоциацию.

Химические соединения, способные к электролитической диссоциации называют \textbf{электролитами}.

Различают \textbf{сильные} и \textbf{слабые} электролиты:

\begin{itemize}
	\item \textit{сильные} даже при больших концентрациях диссоциируют практически полностью. Сильные электролиты --- щелочи, многие неорганические кислоты, почти все соли. 
	\item \textit{слабые} даже в сильно разбавленном растворе имеют какое-то количество недиссоциированных частиц, сравнимое с количеством распавшихся. Слабые электролиты --- многие органические и другие слабые кислоты, большинство оснований, некоторые соли (\ce{ZnCl2}, \ce{HgCl2}, \ce{Hg(CN)2}, \ce{Fe(CNS)3}). 
\end{itemize}

Деление условно, так как всё опять зависит от концентрации.

Для диссоциации можно записать следующие ревновесия:

\begin{equation*}
\ce{M_pX_q = pM^+ + qX^-}.
\end{equation*}

\textbf{Константа диссоциации} --- константа равновесия для этого процесса:

\begin{equation*}
K_D = \frac{c_{+}^{p}c_{-}^{q}}{c},
\end{equation*}

где $c_{+}, c_{-}$ --- концентрации ионов в растворе, $c$ - концентрация молекул в нем же. В числителе константы стоит \textit{произведение растворимости}.\\

\textbf{Степень диссоциации} --- отношение числа продиссоциировавших молекул к полному числу молекул в отсутствие дисссоциации:

$$ \alpha = \frac{c_{+}/p}{c + c_{+}/p} = \frac{1}{1 + pc/c_{+}}. $$

То же можно записать и с $c_{-}/q$. Выразим константу диссоциации через её степень:

$$K_D = \frac{\alpha^2}{1 - \alpha} c_0,$$

где $c_0$ --- общая концентрация электролита в растворе.

\subsection{Диссоциация кислот и оснований}

Для кислоты \ce{HA} уравнение диссоциации запишется в следующем виде:

\begin{equation*}
\ce{HA <--> H^+ + A^-,}
\end{equation*}

где \ce{A^-} --- \textit{сопряженное основание}. \\

\textbf{Константа диссоциации кислоты}:

\begin{equation*}
\ce{K_a = \frac{[H^+][A^-]}{[HA]},}
\end{equation*}

где квадратные скобки --- условное обозначение концентрации. Чем меньше $K_a$, тем слабее кислота. Если кислота диссоциирует в несколько ступеней (на каждой отрывается водород), то константа полной диссоциации есть произведение констант диссоциации каждой ступени.\\

Рассматривают ещё отрицательный логарифм:

$$pK_a = -\lg{K_a}$$

Аналогично для \textbf{оснований} \ce{B}:
\begin{equation*}
\ce{B + H2O <--> BH^+ + OH^-,} \ \ce{K_b = \frac{[BH^+][OH^-]}{[B]},} \ pK_b = -\lg{K_b}
\end{equation*}

Для \textit{слабых кислот и оснований} верно:

\begin{equation*}
\ce{K_a \approx \frac{[H^+]^2}{C - [H^+]} \approx \frac{[H^+]^2}{C} ,\ pH} \approx \frac{1}{2}(pK_a - \lg{C}),
\end{equation*}
\begin{equation*}
\ce{K_b \approx \frac{[OH^-]^2}{C - [OH^-]} \approx \frac{[OH^-]^2}{C} ,\ pH} \approx 14 - \frac{1}{2}(pK_b - \lg{C}),
\end{equation*}
где $C$ --- молярная концентрация вещества.

\subsection{Диссоциация солей}

Соли при диссоциации распадаются на катионы металлов (а также катион аммония \ce{NH4^+}) и анионы кислотных остатков. Примеры уравнений диссоциации:

\begin{equation*}
\ce{Na3PO4 <--> 3Na^+ + PO4^{3-}}
\end{equation*}

\begin{equation*}
\ce{(NH4)_2SO4 <--> 2NH4^+ + SO4^{2-}}
\end{equation*}

Заряды кислотных остатков смотрим в \textit{таблице растворимости}.
