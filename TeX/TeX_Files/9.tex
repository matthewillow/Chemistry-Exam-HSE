\section{Классификация химических реакций. Стехиометрическое описание химической реакции. Энергетическая кривая элементарной химической реакции. Прямая и обратная реакции.}
\subsection{Химические реакции и их классификации}
\textbf{Химическая реакция}  -- превращение одного или нескольких исходных веществ (\textbf{реагентов}) в другие вещества (\textbf{продукты}), при котором ядра атомов не меняются, при этом происходит перераспределение электронов и ядер, и образуются новые химические вещества.

Классификацию химических реакций можно проводить по разным параметрам. Далее приведено несколько варииантов.
\begin{itemize}
    \item По типу превращений реагирующих частиц: 
    \begin{enumerate}
        \item \textbf{Реакция соединения} -- химическая реакция, в результате которой из двух или большего числа исходных веществ образуется только одно новое. В такие реакции могут вступать как простые, так и сложные вещества. Примером может служить окисление лития с получением оксида лития:
        \begin{equation*}
            \ce{4Li + O2 -> 2Li2O}
        \end{equation*}
        \item \textbf{Реакция разложения} -- химическая реакция, в результате которой из одного вещества образуется несколько новых веществ. В реакции данного типа вступают только сложные соединения, а их продуктами могут быть как сложные, так и простые вещества. Примером может служить разложение угольной кислоты на воду и углекислый газ:
        \begin{equation*}
            \ce{H2CO3 -> H2O + CO2 ^}
        \end{equation*}
        \item \textbf{Реакция замещения} -- химическая реакция, в результате которой атомы одного элемента, входящие в состав простого вещества, замещают атомы другого элемента в его сложном соединении. Как следует из определения, в таких реакциях одно из исходных веществ должно быть простым, а другое сложным. Примером может послужить реакция лития (простое вещество) с водой (сложное), с получением гидроксида лития (сложное вещество) и водорода (простое):
        \begin{equation*}
            \ce{2Li + 2H2O -> 2LiOH + H2 ^}
        \end{equation*}
        \item \textbf{Реакции обмена} — реакция, в результате которой два сложных вещества обмениваются своими составными частями. Примером может послужить рассмотренная ранее реакция нейтрализации соляной кислоты и гидроксида натрия с получением поваренной соли и воды:
\begin{equation}
    \ce{HCl + NaOH -> NaCl + H2O}
\end{equation}
    \end{enumerate}
    \item По направлению протекания:
    \begin{enumerate}
        \item \textbf{Необратимые} химические реакции, <<протекающие лишь в одном направлении>>  в том смысле, что реагенты реагируют с получением продуктов, но продукты реакции не реагируют с получением реагентов.  О таких химических процессах говорят, что они протекают «до конца». К ним относят реакции горения, а также реакции, сопровождающиеся образованием малорастворимых или газообразных веществ. Примером может послужить горение углерода (с получением углекислого газа):
        \begin{equation*}
            \ce{C + O2 -> CO2 ^}
        \end{equation*}
        \item \textbf{Обратимыми} называются химические реакции, <<протекающие одновременно в двух противоположных направлениях>>. В уравнениях таких реакций знак равенства заменяется двумя противоположно направленными стрелками. Среди двух одновременно протекающих реакций различают прямую (протекает «слева направо») и обратную (протекает «справа налево»). Поскольку в ходе обратимой реакции исходные вещества одновременно и расходуются, и образуются, они не полностью превращаются в продукты реакции. Поэтому об обратимых реакциях говорят, что они протекают «не до конца». В результате всегда образуется смесь исходных веществ и продуктов взаимодействия. Примером может послужить гидролиз (то есть реакция с водой) нитрита натрия:
        \begin{equation*}
            \ce{NaNO2 + H2O <--> HNO2 + NaOH}
        \end{equation*}
    \end{enumerate}
    \item По тепловому эффекту:
    \begin{enumerate}
        \item \textbf{Экзотермические реакции} -- реакции, которые идут с выделением тепла, (положительный тепловой эффект) например, реакции горения. Так же показательным примером являются реакции активных металлов с водой, далее дана такая реакция для натрия:
        \begin{equation*}
            \ce{2Na + 2H2O -> 2NaOH + H_{2} ^} + Q
        \end{equation*}
        $Q$ -- выделевшееся тепло.
        \item \textbf{Эндотермические реакции} в ходе которых тепло поглощается (отрицательный тепловой эффект) из окружающей среды. Характерный пример разложение карбоната кальция на оксид кальция и углекислый газ при нагреве:
        \begin{equation*}
            \ce{CaCO3 ->[t^{\circ}] CaO + CO2 ^}
        \end{equation*}
        \end{enumerate}
\end{itemize}
Так же реакции можно классифицировать по агрегатному состоянию реагентов и продуктов, по наличию каталлизатора или ингибитора или самопроизвольности (по изменению энергии Гиббса). Выделяют важнейший класс реакций, в которых атомы одного элемента (\textbf{окислителя}) восстанавливаются, то есть присоединяют электроны и понижают свою степень окисления, а атомы другого элемента (\textbf{восстановителя}) окисляются, то есть отдают электроны и повышают свою степень окисления. Такие реакции называются \textbf{окислительно-восстановительными} (сокращённо ОВР).
\subsection{Стехиометрическое описание химической реакции.}
Стехиометрия -- система законов, правил и терминов, обосновывающих расчёты состава веществ и количественных [относительных] соотношений между массами (объёмами для газов) веществ в химических реакциях. Стехиометрия включает нахождение химических формул, составление уравнений химических реакций, расчёты, применяемые в препаративной химии и химическом анализе.

Для большинства реакций можно записать химическое уравнение, в правой части которого стоят реагенты, а в левой -- продукты (вместо знака равно между ними обычно пишут значок \ce{->}). Большинство уравнений реакций написанных в этом конспекте используют стехиометрические правила.  Стехиометрическое уравнение — уравнение, показывающее количественные соотношения реагентов и продуктов химической реакции. Общий вид стехиометрического уравнения химической реакции таков:

\begin{equation}
    \sum_{i = 1}^{N}n_{i}X_{i} = \sum_{j = 1}^{M}m_{j}Y_{i}, \ \text{где} \ N,M,n_{i},m_{i}\in\mathbb{N}
\end{equation}
Натуральные числа $n_{i}$ и $m_{j}$ называются стехиометрическими коэффициентами. Эта запись означает, что  $n_{1}$ молекул реагента $X_{1}$, $n_{2}$ молекул реагента $X_{2}$, …, $n_{N}$ молекул реагента $X_{N}$, вступив в реакцию, образуют $m_{1}$ молекул вещества $Y_{1}$, $m_2$ молекул вещества $Y_{2}$, …,  $m_{M}$ молекул вещества $Y_{M}$. 

Коэффициенты $n_{i}$ и $m_{j}$ определяются так, чтобы количества атомов одного и того же вещества и справа и слева оставалось одним и тем же (тем самым удовлетворяется \textit{химичность} реакции (т.к. элементы не изменяются), а так же выполняется закон сохранения массы). Расчёт коэффициентов $n_{i}$ и $m_{j}$ для данных реагентов и продуктов называется уравниванием коэффициентов, и может производиться разными методами, например подбором с учётом валентностей, однако реакции с несколькими сложными реагентами и продуктами уравнять таким образом довольно сложно, поэтому используются другие методы (например электронный баланс в случае ОВР).
\subsection{Энергетическая кривая химической реакции. }
\begin{figure}[H]
\centering
\begin{tikzpicture}[>=latex']
    \draw[domain=-3:3.5, samples=100, red, thick] plot (\x,{(1/2 + abs(\x)/(2*\x))*(3*exp(- \x*\x)) + (1/2 - abs(\x)/(2*\x))*(1 + 2*exp(- \x*\x))});
    \draw[dashed] (-3.5,1) -- (0.5,1);
    \draw[dashed] (3.5,0) -- (-0.5,0);
    \draw[dashed] (-3.5,3) -- (2.75,3);
    \draw[<->,blue] (0,0) -- (0,1) node[left, pos=0.5, black] {$\Delta H$};
    \draw[->] (-3,-0.8) -- (-3,3.4) node[above]{$E$};
    \draw[->] (-3.5,-0.5) -- (3.7,-0.5) node[right]{$\xi$};
    \draw[<->,blue] (-2.5,1) -- (-2.5,3) node[right, pos=0.5, black]{$\Delta E_{\longrightarrow}$};
    \draw[<->,blue] (2.5,0) -- (2.5,3) node[right, pos=0.5, black]{$\Delta E_{\longleftarrow}$};
\end{tikzpicture}
\caption{Энергетическая кривая. $E_{\longrightarrow}$ -- энергия активации прямой реакции, $E_{\longleftarrow}$ -- энергия активации обратной реакции.}
\label{fig:energycurve}
\end{figure}
Точное определение химической переменной $\xi_{i}$ может быть задано так:
\begin{equation}
    \xi_{i} = \frac{\mathrm{d}\mathcal{V}_{i}}{\mathrm{d}n_{i}} = \frac{\Delta \mathcal{V}_{i}}{\Delta n_{i}} = \pm\frac{\Delta \mathcal{V}_{i}}{n_{i}}
\end{equation}
Здесь $\mathcal{V}_{i}$ -- количество молей $i$-того реагента (для продукта координата определяется так же только с минусом), а $n_{i}$ -- его стехиометрический коэффициент в уравнении рассматриваемой реакции. Изменение $\Delta$ подразумевает разницу между началом реакции и каким-то моментом её протекания. Так как $n_{i}$ входит только  в одну из сторон уравнения, то $\Delta n_{i} = \pm n_{i} $, где плюс выбирается для реагента, а минус для продукта реакции.

Для химической реакции или процесса энергетическая кривая  является теоретическим представлением единственного энергетического пути вдоль координаты реакции, когда реагенты превращаются в продукты. Диаграммы координат реакций выводятся из соответствующей поверхности потенциальной энергии (ППЭ) (так как таких обобщённых координат у любой кривой вообще говоря несколько), которые используются в вычислительной химии для моделирования химических реакций, связывая энергию молекулы (молекул) с ее структурой (в рамках приближения Борна – Оппенгеймера ). Координата реакции - это параметрическая кривая, которая следует за ходом реакции и показывает ее ход.

На рисунке~\ref{fig:energycurve} привидён пример энергетической кривой элементарной реакции\footnote{Элементарна реакция в том смысле, что у графика один пик, соответствующий прямому переходу из реагента в продукт. Большее количество пиков соответствовало бы нескольким промежуточным продуктам.}. Так как $\Delta E_{\longrightarrow} < \Delta E_{\longleftarrow}$ прямая реакция является экзотермической, а обратная -- эндотермическая, при этом высвобождаемая или сообщаемая теплота равна $\Delta H$. Внесение катализатора соответствовало бы снижению горба кривой, а ингибитора наоборот -- завышению, то есть высота горба характерезует скорость протечения реакции.
